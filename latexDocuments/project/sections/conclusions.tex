\section{Final Thoughts}
\label{sec:Conclu}

Something we learned when building data visualizations for climate data is that you are always trying to balance simplicity of the approach with closeness to the phenomena.  Ed Hawkins, climate scientist from Reading University created the warming stripes visualization which you can see Colombia's version atop of the web application.

They show a clear picture of global warming since the 1900 in such an elegant way that they went viral and did a great service in terms of raising awareness. We took inspiration from that but decided to go into more detail in order to show how global warming affects you directly. The town you were born, the city you live.

Hopefully we accomplished some of that: Surfacing information from the past, removing friction to access current data and giving some clues about the future. This was the \textbf{challenge accepted} by 6 college educated professionals, experienced in the fields of data analytics, information management and research to put all the data together, in a way that any of us, as citizens, could make a decision, derive a conclusion or even form an opinion upon it.

Finally, to accomplish what was shown in this document, the work was divided as follows:

\begin{itemize}

\item Database AWS Computation / SQL queries: Alvaro Mu\~noz and Mario Cer\'on
\item Front End Dash App: Javier Coconubo, Jhonatan Salamanca and Jairo Ni\~no
\item Data Wrangling: Javier Coconubo, Carol Martinez, Jhonatan Salamanca and Alvaro Mu\~noz.
\item Dataset Research: Jairo Ni\~no and Carol Martinez
\item Project Management: Mario Cer\'on and Javier Coconubo
\item Documentation and reporting: Carol Martinez and Mario Cer\'on

\end{itemize}

All of the 6 members of the group collaborate in equal efforts as well. ($16,6\%$) 
