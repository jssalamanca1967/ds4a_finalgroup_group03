\section{Data Wrangling and Data Cleaning}
\label{subsec:dataCl}

The data cleaning process was done in two steps:

\begin{itemize}
\item For yellow and green cab trips, the rows that have distances equal to $0$ were deleted. This, because we are aiming to take into account only the trips that traveled some distance.
\item For yellow and green cab trips, the IQR (Inter Quartile Range) methodology was used to clean the outliers from the data. A variable called ``amount\_per\_distance'' was created. It was calculated as the ratio between ``total\_amount" and ``trip\_distance". With this new variable, the values that did not show a common relationship between distance and values were deleted.
%\item The borough polygons were used XXX
\item When analyzing the data, we encountered that the columns precipitation, snowfall and snow\_depth had missing values in the form of a ? ? character. For each column, we found 237 ($10.82\%$), 91 ($4.15\%$), 24 ($1.09\%$) empty values respectively. Considering that these variables are highly correlated with the average temperature, we decided to apply an iterative imputation with a decision tree regressor estimator to them.
\end{itemize}

\begin{table}[h]
\begin{center}
\label{tab:dataset}
\begin{tabular}{|c|c|c|c|c|}
\hline
\textbf{Dataset} & \textbf{Initial} & \textbf{Deleted} & \textbf{Final}  \\
\hline
 Uber trips &  18676106	& 0  & ss\\
Yellow cab trips & 7926168 & 337998 & 7588770\\
Green cab trips &  3537586 & 186494 & 3351092\\
MTA trips & 7554197	&  0	& ss\\
Weather & 	2190 &  0 & 2190\\
\hline
\end{tabular}
 \caption{Summary of the main information available to develop the project.}
\end{center}
\end{table}


Feature engineering:
We created a new variable that measures the ratio between the total amount of the trip and the distance it traveled. This feature was created for Yellow trips and Green trips and was used for the outlier cleansing.

